\documentclass[11pt]{article}
\usepackage{algorithm2e}
\usepackage[italian]{babel}
\usepackage[document]{ragged2e}
\usepackage{amsfonts, amssymb, amsmath}
\usepackage{cancel}
\usepackage{float}
\usepackage{mathtools}
\usepackage[margin=3cm]{geometry}
\usepackage{subfig}
\usepackage{mwe}
\usepackage{hyperref}

\usepackage[
backend=biber,
sorting=none
]{biblatex}
\addbibresource{bibliografia.bib}

\tolerance=1
\emergencystretch=\maxdimen
\hyphenpenalty=10000
\hbadness=10000

\begin{document}
\graphicspath{ {./img/} }
\begin{titlepage}
    \begin{center}
        \vspace*{1.5cm}
            
        \Huge
        \textbf{PandOS Plus} \\
        \LARGE
        Fase 1
                        
        \vspace{2.0cm}
          
        \begin{minipage}[t]{0.47\textwidth}
        \begin{center}
        	{\large{\bf Cheikh Ibrahim $\cdot$ Zaid}}\\
			{\large Matricola: \texttt{0000974909}}
        \end{center}

		\end{minipage}
		\hfill
		\begin{minipage}[t]{0.47\textwidth}\raggedleft
		\begin{center}
        	{\large{\bf Lee $\cdot$ Qun Hao Henry}}\\
			{\large Matricola: \texttt{0000990259}}
        \end{center}
		\end{minipage}

        \vspace{1cm}

        \begin{minipage}[t]{0.47\textwidth}
            \begin{center}
                {\large{\bf Xia $\cdot$ Tian Cheng}}\\
                {\large Matricola: \texttt{0000975129}}
            \end{center}
    
            \end{minipage}
            \hfill
            \begin{minipage}[t]{0.47\textwidth}\raggedleft
            \begin{center}
                {\large{\bf Paris $\cdot$ Manuel}}\\
                {\large Matricola: \texttt{0000997526}}
            \end{center}
            \end{minipage}  
            
        \vspace{6cm}
            
        Anno accademico\\
        $2021 - 2022$
            
        \vspace{0.8cm}
            
            
        \Large
        Corso di Sistemi Operativi\\
        Alma Mater Studiorum $\cdot$ Università di Bologna\\
            
    \end{center}
\end{titlepage}
\pagebreak

\newpage

\section{Introduzione}
La prima fase del progetto \texttt{PandOS+} consiste nell'implementazione del livello 2 dell'architettura di astrazione proposta da Dijkstra.\\
In particolare è necessario implementare le strutture dati per la gestione di PCB e semafori.

\section{Gestione di PCB}
Tutto secondo le specifiche. Grazie. Ciao.

\section{Gestione semafori}
\subsection{Lista dei semafori liberi}
La lista dei semafori liberi è stata implementata utilizzando una stack con politica FIFO. \\
Tale scelta è stata fatta ipotizzando che nel processore sia presente un meccanismo di cache della memoria che mantenga salvati gli indirizzi più recenti. 
In questo modo, nel momento in cui un semaforo viene disimpiegato, 
lo stesso sarà il primo ad essere estratto se successivamente fosse necessario istanziare un semaforo, aumentando le possibilità di trovarlo all'interno della cache.

\subsection{Lista dei semafori attivi}
La lista dei semafori liberi (ASL) viene gestita utilizzando una lista ordinata rispetto alla chiave del semaforo come suggerito dallo \textit{Student Guide} \textbf{[AAA RICORDARSI DI METTERE IL RIFERIMENTO]}. \\
Poiché si utilizzano liste bidirezionali, è stata realizzata una variazione rispetto all'implementazione proposta che non utilizza due nodi ausiliari ma 
esegue un controllo per gestire il caso particolare in cui l'inserimento debba avvenire in coda.\\
L'implementazione con lista ordinata permette di ottenere una soluzione più efficiente nella ricerca del semaforo attivo, riducendo il tempo di ricerca dei semafori attivi.
\subsubsection{Possibili alternative}
Coso che memorizza il mezzo

\newpage
\printbibliography[title={Bibliografia}]

\end{document}
