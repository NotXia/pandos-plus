\documentclass[11pt]{article}
\usepackage{algorithm2e}
\usepackage[italian]{babel}
\usepackage[document]{ragged2e}
\usepackage{amsfonts, amssymb, amsmath}
\usepackage{cancel}
\usepackage{float}
\usepackage{mathtools}
\usepackage[margin=3cm]{geometry}
\usepackage{subfig}
\usepackage{mwe}
\usepackage{hyperref}

\usepackage[
backend=biber,
sorting=none
]{biblatex}
\addbibresource{bibliografia.bib}

\tolerance=1
\emergencystretch=\maxdimen
\hyphenpenalty=10000
\hbadness=10000

\begin{document}
\graphicspath{ {./img/} }
\begin{titlepage}
    \begin{center}
        \vspace*{1.5cm}
            
        \Huge
        \textbf{PandOS}\texttt{+} \\
        \LARGE
        Fase 1
                        
        \vspace{2.0cm}
          
        \begin{minipage}[t]{0.47\textwidth}
        \begin{center}
        	{\large{\bf Cheikh Ibrahim $\cdot$ Zaid}}\\
			{\large Matricola: \texttt{0000974909}}
        \end{center}

		\end{minipage}
		\hfill
		\begin{minipage}[t]{0.47\textwidth}\raggedleft
		\begin{center}
        	{\large{\bf Lee $\cdot$ Qun Hao Henry}}\\
			{\large Matricola: \texttt{0000990259}}
        \end{center}
		\end{minipage}

        \vspace{1cm}

        \begin{minipage}[t]{0.47\textwidth}
            \begin{center}
                {\large{\bf Xia $\cdot$ Tian Cheng}}\\
                {\large Matricola: \texttt{0000975129}}
            \end{center}
    
            \end{minipage}
            \hfill
            \begin{minipage}[t]{0.47\textwidth}\raggedleft
            \begin{center}
                {\large{\bf Paris $\cdot$ Manuel}}\\
                {\large Matricola: \texttt{0000997526}}
            \end{center}
            \end{minipage}  
            
        \vspace{6cm}
            
        Anno	 accademico\\
        $2021 - 2022$
            
        \vspace{0.8cm}
            
            
        \Large
        Corso di Sistemi Operativi\\
        Alma Mater Studiorum $\cdot$ Università di Bologna\\
            
    \end{center}
\end{titlepage}
\pagebreak

\newpage

\section{Introduzione}
La prima fase del progetto \texttt{PandOS+} consiste nell'implementazione del livello 2 dell'architettura astratta di un sistema operativo proposta da Dijkstra.\\
In particolare è necessario implementare le strutture dati per la gestione di PCB e semafori.

\section{Gestione di PCB}
La gestione dei PCB è stata implementata secondo le specifiche proposte. \\
L'unico aspetto per cui è stata fatta una scelta particolare è la gestione della lista dei PCB liberi, per la quale è stata adottata una struttura dati di tipo stack con politica LIFO.\\
Tale decisione è stata fatta ipotizzando che nell'architettura sia presente un meccanismo di caching della memoria. In questo modo quando un processo termina, 
il relativo PCB, che avrà maggiore probabilità di essere ancora in cache in un momento successivo, sarà il primo ad essere selezionato per istanziare un nuovo processo.

\section{Gestione dei semafori}
\subsection{Lista dei semafori liberi}
La lista dei semafori liberi è stata implementata con lo stesso approccio adottato precedentemente, utilizzando una stack con politica LIFO. \\
Anche in questo caso, l'obiettivo è quello di aumentare le probabilità di una cache hit, inizializzando in modo più rapido un semaforo nuovo a seguito del disimpiego di un altro.

\subsection{Lista dei semafori attivi}
La lista dei semafori liberi (ASL) viene gestita utilizzando una lista ordinata rispetto alla chiave del semaforo, come proposto dallo \textit{Student Guide} \cite{1}. \\
Rispetto all'implementazione suggerita, è stata realizzata una variazione che non utilizza due nodi ausiliari ma esegue un controllo per gestire i casi particolari in cui l'inserimento debba avvenire in coda o con lista vuota.\\
Il fine di questa implementazione è quello di ottenere una soluzione in grado di ridurre il tempo di ricerca del semaforo attivo, a discapito di un inserimento più lento.
Tale approccio è accettabile se si suppone che il numero di operazioni sui semafori attivi sia maggiore rispetto alle inizializzazioni di semafori nuovi.

\newpage
\printbibliography[title={Bibliografia}]

\end{document}
 