\documentclass[11pt]{article}
\usepackage{algorithm2e}
\usepackage[italian]{babel}
\usepackage[document]{ragged2e}
\usepackage{amsfonts, amssymb, amsmath}
\usepackage{cancel}
\usepackage{float}
\usepackage{mathtools}
\usepackage[margin=3cm]{geometry}
\usepackage{subfig}
\usepackage{mwe}
\usepackage{hyperref}
\usepackage{array}

\usepackage[
backend=biber,
sorting=none
]{biblatex}
\addbibresource{bibliografia.bib}

\tolerance=1
\emergencystretch=\maxdimen
\hyphenpenalty=10000
\hbadness=10000

\begin{document}
\graphicspath{ {./img/} }
\begin{titlepage}
    \begin{center}
        \vspace*{1.5cm}
            
        \Huge
        \textbf{PandOS}\texttt{+} \\
        \LARGE
        Fase 2
                        
        \vspace{2.0cm}
          
        \begin{minipage}[t]{0.47\textwidth}
        \begin{center}
        	{\large{\bf Cheikh Ibrahim $\cdot$ Zaid}}\\
			{\large Matricola: \texttt{0000974909}}
        \end{center}

		\end{minipage}
		\hfill
		\begin{minipage}[t]{0.47\textwidth}\raggedleft
		\begin{center}
        	{\large{\bf Lee $\cdot$ Qun Hao Henry}}\\
			{\large Matricola: \texttt{0000990259}}
        \end{center}
		\end{minipage}

        \vspace{1cm}

        \begin{minipage}[t]{0.47\textwidth}
            \begin{center}
                {\large{\bf Xia $\cdot$ Tian Cheng}}\\
                {\large Matricola: \texttt{0000975129}}
            \end{center}
    
            \end{minipage}
            \hfill
            \begin{minipage}[t]{0.47\textwidth}\raggedleft
            \begin{center}
                {\large{\bf Paris $\cdot$ Manuel}}\\
                {\large Matricola: \texttt{0000997526}}
            \end{center}
            \end{minipage}  
            
        \vspace{6cm}
            
        Anno	 accademico\\
        $2021 - 2022$
            
        \vspace{0.8cm}
            
            
        \Large
        Corso di Sistemi Operativi\\
        Alma Mater Studiorum $\cdot$ Università di Bologna\\
            
    \end{center}
\end{titlepage}
\pagebreak

\newpage

\section{Introduzione}
La seconda fase del progetto \texttt{PandOS+} consiste nell'implementazione del livello 3 dell'architettura astratta di un sistema operativo proposta da Dijkstra. \\
In particolare è necessario implementare le funzionalità del kernel per gestire la schedulazione dei processi e la gestione delle eccezioni.

\subsection{Organizzazione dei file}
I file coinvolti per l'implementazione delle specifiche sono i seguenti:
\begin{center}
    \begin{tabular}{ | m{3cm} | m{10cm} | } 
        \hline
        \texttt{pcb}          & È stata aggiunta la gestione dei \texttt{pid} dei processi. \\ 
        \hline
        \texttt{asl}          & Sono state aggiunte le operazioni \texttt{P}/\texttt{V} sui semafori. \\ 
        \hline
        \texttt{initial}      & Gestisce la fase di inizializzazione del sistema e fornisce alcune funzioni ausiliare di utilità generale. \\ 
        \hline
        \texttt{scheduler}    & Implementa lo scheduler. \\ 
        \hline
        \texttt{exceptions}   & Gestisce lo "smistamento" le eccezioni ai relativi gestori. Implementa il gestore delle system call. \\ 
        \hline
        \texttt{interrupts}   & Implementa il gestore delle interrupt. \\ 
        \hline
        \texttt{utilities}    & Contiene l'implementazione di \texttt{memcpy}. \\ 
        \hline
    \end{tabular}
\end{center}

\section{Gestione dei \texttt{pid}}
I pid vengono generati in ordine crescente a partire da 1.\\
La mappatura dei pid validi in un dato istante avviene tramite una lista bidirezionale implementata attraverso un nuovo campo aggiunto della struttura \texttt{pcb\_t}.
Tale lista è ordinata in senso crescente per migliorare le prestazioni di ricerca e permette l'inserimento in tempo costante per il primo ciclo di pid (dopo un wraparound non è garantita questa proprietà).
\subsection{Possibili miglioramenti}
Una possibile alternativa a tale implementazione è basata sugli alberi di ricerca che permettono di ottenere maggiori prestazioni nella fase di ricerca. \\
La scelta di utilizzare una lista ordinata è stata fatta considerando un trade-off tra semplicità di implementazione e prestazioni che rimangono valide in quanto il numero massimo di processi attivi ha un limite superiore molto contenuto.

\section{Gestione dei semafori dei device}


\section{Gestione contabilizzazione del tempo di CPU}


\section{Inizializzazione del sistema}


\section{Scheduling dei processi}


\section{Gestione delle eccezioni}


\newpage
% \printbibliography[title={Bibliografia}]

\end{document}
 