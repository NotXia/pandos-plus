\documentclass[11pt]{article}
\usepackage{algorithm2e}
\usepackage[italian]{babel}
\usepackage[document]{ragged2e}
\usepackage{amsfonts, amssymb, amsmath}
\usepackage{cancel}
\usepackage{float}
\usepackage{mathtools}
\usepackage[margin=3cm]{geometry}
\usepackage{subfig}
\usepackage{mwe}
\usepackage{hyperref}
\usepackage{array}

\usepackage[
backend=biber,
sorting=none
]{biblatex}
\addbibresource{bibliografia.bib}

\tolerance=1
\emergencystretch=\maxdimen
\hyphenpenalty=10000
\hbadness=10000

\begin{document}
\graphicspath{ {./img/} }
\begin{titlepage}
    \begin{center}
        \vspace*{1.5cm}
            
        \Huge
        \textbf{PandOS}\texttt{+} \\
        \LARGE
        Fase 2
                        
        \vspace{2.0cm}
          
        \begin{minipage}[t]{0.47\textwidth}
        \begin{center}
        	{\large{\bf Cheikh Ibrahim $\cdot$ Zaid}}\\
			{\large Matricola: \texttt{0000974909}}
        \end{center}

		\end{minipage}
		\hfill
		\begin{minipage}[t]{0.47\textwidth}\raggedleft
		\begin{center}
        	{\large{\bf Lee $\cdot$ Qun Hao Henry}}\\
			{\large Matricola: \texttt{0000990259}}
        \end{center}
		\end{minipage}

        \vspace{1cm}

        \begin{minipage}[t]{0.47\textwidth}
            \begin{center}
                {\large{\bf Xia $\cdot$ Tian Cheng}}\\
                {\large Matricola: \texttt{0000975129}}
            \end{center}
    
            \end{minipage}
            \hfill
            \begin{minipage}[t]{0.47\textwidth}\raggedleft
            \begin{center}
                {\large{\bf Paris $\cdot$ Manuel}}\\
                {\large Matricola: \texttt{0000997526}}
            \end{center}
            \end{minipage}  
            
        \vspace{6cm}
            
        Anno	 accademico\\
        $2021 - 2022$
            
        \vspace{0.8cm}
            
            
        \Large
        Corso di Sistemi Operativi\\
        Alma Mater Studiorum $\cdot$ Università di Bologna\\
            
    \end{center}
\end{titlepage}
\pagebreak

\newpage

\section{Introduzione}
La seconda fase del progetto \texttt{PandOS+} consiste nell'implementazione del livello 3 dell'architettura astratta di un sistema operativo proposta da Dijkstra. \\
In particolare è necessario implementare le funzionalità del kernel per l'inizializzazione del sistema, la schedulazione dei processi e la gestione delle eccezioni.

\subsection{Organizzazione dei file}
I file coinvolti per l'implementazione delle specifiche sono i seguenti:
\begin{center}
    \begin{tabular}{ | m{3cm} | m{10cm} | } 
        \hline
        \texttt{pcb}          & È stata aggiunta la gestione dei \texttt{pid} dei processi. \\ 
        \hline
        \texttt{asl}          & Sono state aggiunte le operazioni \texttt{P}/\texttt{V} sui semafori. \\ 
        \hline
        \texttt{initial}      & Gestisce la fase di inizializzazione del sistema e fornisce alcune funzioni ausiliare di utilità generale. \\ 
        \hline
        \texttt{scheduler}    & Implementa lo scheduler. \\ 
        \hline
        \texttt{exceptions}   & Reindirizza le eccezioni ai relativi gestori.\\
        Implementa il gestore delle system call. \\ 
        \hline
        \texttt{interrupts}   & Implementa il gestore degli interrupt. \\ 
        \hline
        \texttt{utilities}    & Contiene l'implementazione di \texttt{memcpy}. \\ 
        \hline
    \end{tabular}
\end{center}

\section{Gestione dei \texttt{pid}}
I pid vengono generati in modo incrementale a partire da 1.\\
La mappatura dei pid validi in un dato istante avviene tramite una lista bidirezionale implementata attraverso un nuovo campo aggiunto della struttura \texttt{pcb\_t}.
Tale lista è ordinata in senso crescente per migliorare le prestazioni di ricerca e permette l'inserimento in tempo costante per il primo "ciclo di pid" (dopo un wraparound non è garantita questa proprietà).
\subsection{Possibili miglioramenti}
Sono possibili alternative a tale implementazione utilizzando strutture dati con approcci differenti che permettono di ottenere prestazioni diverse (es. alberi di ricerca oppure utilizzando direttamente l'albero dei processi). \\
La scelta di utilizzare una lista ordinata è stata fatta considerando un trade-off tra semplicità di implementazione e prestazioni, che rimangono comunque valide in quanto il numero massimo di processi attivi ha un limite superiore molto contenuto e l'overhead di memoria per mantenere la lista non è eccessivo.

\newpage
\section{Gestione dei semafori per i device}
\subsection{Interval timer}
L'interval timer ha un semaforo dedicato e il tempo caricato ad ogni ciclo viene calcolato in base alla seguente formula:
\begin{equation}
    \text{time}_{\text{IT}} = \texttt{PSECOND} - (\text{curr}\_\text{time } \% \texttt{ PSECOND})
\end{equation}
Dove \texttt{PSECOND} è la durata dell'interval timer e $\text{curr}\_\text{time}$ è il valore del \texttt{TOD}.\\
In questo modo, il timer viene impostato facendo riferimento al momento di inizio del \texttt{TOD}, permettendo di non accumulare eventuali ritardi causati dalla gestione dell'interrupt.

\subsection{Device di I/O}
I semafori associati ai device di I/O sono memorizzati in un vettore la cui mappatura tra indice e device segue la seguente formula:
\begin{equation}
    i = \dfrac{\text{dev\_reg}_{\text{address}} - \text{dev\_reg}_{\text{start\_address}}}{\text{dev\_reg}_{\text{size}}} 
\end{equation}
Dove:
\begin{itemize}
    \item $\text{dev\_reg}_{\text{address}}$ è il parametro che indica l'indirizzo del device register associato al device per cui si vuole trovare il semaforo
        \footnote{Durante l'elaborazione, in alcuni casi viene fornito l'indirizzo del campo \textit{command} del device register. A partire da tale indirizzo è comunque possibile risalire all'indirizzo di inizio del device register (bisogna gestire il fatto che i terminali hanno due campi \textit{command}).}.
    \item $\text{dev\_reg}_{\text{start\_address}}=0\text{x}10000054$ è l'indirizzo da cui iniziano i device register.
    \item $\text{dev\_reg}_{\text{size}}=0\text{x}10$ è la dimensione dei device register (4 word).
\end{itemize}
Quindi al primo indice del vettore si trova il semaforo associato al device di I/O con priorità maggiore (disco 1) e all'ultimo quello con priorità minore (trasmissione terminale 7).


\section{Gestione del tempo di CPU}
Il calcolo del tempo di CPU per ciascun processo utilizza, come da specifiche, il meccanismo di temporizzazione fornito dal \texttt{TOD}.\\
La politica di accumulo prevede di contabilizzare, per ciascun processo, il tempo di esecuzione effettivo e il tempo impiegato per eseguire le system call (poiché sono direttamente invocate dal processo).
Non viene invece considerato il tempo di gestione degli interrupt, durante il quale viene disabilitato il \texttt{PLT}, in quanto non sono causate in modo diretto dal processo.

\newpage
% \printbibliography[title={Bibliografia}]

\end{document}