\documentclass[11pt]{article}
\usepackage{algorithm2e}
\usepackage[italian]{babel}
\usepackage[document]{ragged2e}
\usepackage{amsfonts, amssymb, amsmath}
\usepackage{cancel}
\usepackage{float}
\usepackage{mathtools}
\usepackage[margin=3cm]{geometry}
\usepackage{subfig}
\usepackage{mwe}
\usepackage{hyperref}
\usepackage{array}

\usepackage[
backend=biber,
sorting=none
]{biblatex}
\addbibresource{bibliografia.bib}

\tolerance=1
\emergencystretch=\maxdimen
\hyphenpenalty=10000
\hbadness=10000

\begin{document}
\graphicspath{ {./img/} }
\begin{titlepage}
    \begin{center}
        \vspace*{1.5cm}
            
        \Huge
        \textbf{PandOS}\texttt{+} \\
        \LARGE
        Fase 3
                        
        \vspace{2.0cm}
          
        \begin{minipage}[t]{0.47\textwidth}
        \begin{center}
        	{\large{\bf Cheikh Ibrahim $\cdot$ Zaid}}\\
			{\large Matricola: \texttt{0000974909}}
        \end{center}

		\end{minipage}
		\hfill
		\begin{minipage}[t]{0.47\textwidth}\raggedleft
		\begin{center}
        	{\large{\bf Lee $\cdot$ Qun Hao Henry}}\\
			{\large Matricola: \texttt{0000990259}}
        \end{center}
		\end{minipage}

        \vspace{1cm}

        \begin{minipage}[t]{0.47\textwidth}
            \begin{center}
                {\large{\bf Xia $\cdot$ Tian Cheng}}\\
                {\large Matricola: \texttt{0000975129}}
            \end{center}
    
            \end{minipage}
            \hfill
            \begin{minipage}[t]{0.47\textwidth}\raggedleft
            \begin{center}
                {\large{\bf Paris $\cdot$ Manuel}}\\
                {\large Matricola: \texttt{0000997526}}
            \end{center}
            \end{minipage}  
            
        \vspace{6cm}
            
        Anno	 accademico\\
        $2021 - 2022$
            
        \vspace{0.8cm}
            
            
        \Large
        Corso di Sistemi Operativi\\
        Alma Mater Studiorum $\cdot$ Università di Bologna\\
            
    \end{center}
\end{titlepage}
\pagebreak

\newpage

\section{Introduzione}
La terza fase del progetto \texttt{PandOS+} consiste nell'implementazione del livello di supporto, 
in particolare è necessario lo sviluppo della memoria virtuale e di un meccanismo di interfacciamento per i processi utente alle system call del kernel.\\
Per l'implementazione della consegna, sono state seguite le specifiche e le ottimizzazioni proposte. 
Di seguito sono descritte alcune parti del progetto le cui implementazioni non sono esplicitamente presenti nelle specifiche.

\section{Inizializzazione della tabella delle pagine}
Come indicato dalle specifiche, l'inizializzazione della tabella delle pagine di un processo deve marcare le pagine \texttt{.text} del processo come read-only.
Questo richiede di leggere il contenuto dell'header del processo che risiede nella prima pagina. \\
Poiché lo stack del processo \texttt{test} è contenuto in un singolo frame, il caricamento dell'header comporterebbe un inevitabile overflow dello stack che 
causerebbe la sovrascrittura delle aree successive dedicate agli stack di gestione delle eccezioni dei processi utente. \\
La soluzione adottata per risolvere il problema prevede di inizializzare la tabella delle pagine dopo aver assegnato i frame per lo stack al processo utente.
In questo modo è possibile utilizzare temporaneamente i frame dello stack per caricare l'header in quanto il processo non è ancora stato avviato e non necessita ancora di quelle aree di memoria.

\section{Gestione dei semafori al livello di supporto}
I semafori al livello di supporto vengono gestiti tramite una struttura dati che associa al valore corrente l'ASID del processo che ne detiene il controllo.
Questo risulta utile per individuare, in modo più semplice, il processo che deve rilasciare le risorse a seguito di eventi "esterni" che ne causano la terminazione anomala.

\section{Testing}
Per la compilazione dei test, è stato modificato il \texttt{Makefile} (e conseguentemente i sorgenti) seguendo l'esempio \cite{1}. 
Questo è stato necessario in quanto la versione precedente non generava correttamente il percorso alla libreria di $\mu$MPS3.

\newpage
\printbibliography[title={Bibliografia}]

\end{document}