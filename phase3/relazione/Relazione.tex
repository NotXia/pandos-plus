\documentclass[11pt]{article}
\usepackage{algorithm2e}
\usepackage[italian]{babel}
\usepackage[document]{ragged2e}
\usepackage{amsfonts, amssymb, amsmath}
\usepackage{cancel}
\usepackage{float}
\usepackage{mathtools}
\usepackage[margin=3cm]{geometry}
\usepackage{subfig}
\usepackage{mwe}
\usepackage{hyperref}
\usepackage{array}

\usepackage[
backend=biber,
sorting=none
]{biblatex}
\addbibresource{bibliografia.bib}

\tolerance=1
\emergencystretch=\maxdimen
\hyphenpenalty=10000
\hbadness=10000

\begin{document}
\graphicspath{ {./img/} }
\begin{titlepage}
    \begin{center}
        \vspace*{1.5cm}
            
        \Huge
        \textbf{PandOS}\texttt{+} \\
        \LARGE
        Fase 3
                        
        \vspace{2.0cm}
          
        \begin{minipage}[t]{0.47\textwidth}
        \begin{center}
        	{\large{\bf Cheikh Ibrahim $\cdot$ Zaid}}\\
			{\large Matricola: \texttt{0000974909}}
        \end{center}

		\end{minipage}
		\hfill
		\begin{minipage}[t]{0.47\textwidth}\raggedleft
		\begin{center}
        	{\large{\bf Lee $\cdot$ Qun Hao Henry}}\\
			{\large Matricola: \texttt{0000990259}}
        \end{center}
		\end{minipage}

        \vspace{1cm}

        \begin{minipage}[t]{0.47\textwidth}
            \begin{center}
                {\large{\bf Xia $\cdot$ Tian Cheng}}\\
                {\large Matricola: \texttt{0000975129}}
            \end{center}
    
            \end{minipage}
            \hfill
            \begin{minipage}[t]{0.47\textwidth}\raggedleft
            \begin{center}
                {\large{\bf Paris $\cdot$ Manuel}}\\
                {\large Matricola: \texttt{0000997526}}
            \end{center}
            \end{minipage}  
            
        \vspace{6cm}
            
        Anno	 accademico\\
        $2021 - 2022$
            
        \vspace{0.8cm}
            
            
        \Large
        Corso di Sistemi Operativi\\
        Alma Mater Studiorum $\cdot$ Università di Bologna\\
            
    \end{center}
\end{titlepage}
\pagebreak

\newpage

\section{Introduzione}
La terza fase del progetto \texttt{PandOS+} consiste nell'implementazione del livello di supporto, 
in particolare è necessario lo sviluppo della memoria virtuale e di un meccanismo di interfacciamento alle system call che richiedono la modalità kernel.\\


\section{Testing}
Per la compilazione dei test, è stato modificato il \texttt{Makefile} (e conseguentemente i sorgenti) seguendo l'esempio \cite{1}. 
Questo è stato necessario in quanto la versione precedente non generava correttamente il percorso alla libreria di $\mu$MPS3.

% È

\newpage
\printbibliography[title={Bibliografia}]

\end{document}